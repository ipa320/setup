%##########################################################################
%                                                                     
%                         Care-O-bot Manual                                                        
%                                     
%##########################################################################


%##########################################################################
% Formatierungsoptionen
% fuer Bilder die Option "draft" entfernen
\documentclass[12pt,twoside]{report}

% Standard Style-Files
%\usepackage{german}
\usepackage{a4}
%\usepackage{psfig}
\usepackage{graphicx}
\usepackage{subfigure}
%\usepackage{equations}
\usepackage{thb}
\usepackage{amssymb}
\usepackage{listings}
\usepackage{datetime}

% define colors
\usepackage{color}
\definecolor{light-gray}{gray}{0.85}

% Abruerzungsverzeichnis
\usepackage{nomencl}

% Hyperlinks
\usepackage[bookmarksnumbered=true,backref=page,breaklinks=true,pdftitle={IPA robot and ipa-apartment manual},
pdfauthor={Nadia Hammoudeh Garcia, Florian Weisshardt},pdfsubject={IPA robot and ipa-apartment manual},
pdfkeywords={IPA, ipa-apartment, manual}]{hyperref}


% Seitenstil
\pagestyle{headings}
%
% Abstand zwischen Abs"atzen
\setlength{\parskip}{1.5ex}

% Einr"uckung der ersten Zeile eines Absatzes unterdr"ucken
\setlength{\parindent}{0pt}

% Grosszuegigere Wortabstaende
\sloppy

% Tiefe der numerierten Kapitel definieren
\setcounter{secnumdepth}{3}

% Tiefe der Kapitel im Inahltsverzeichnis definieren
\setcounter{tocdepth}{3}

\longdate

% Damit Bilder m"oglichst da sind, wo man sie will
\setcounter{topnumber}{20}
\setcounter{bottomnumber}{20}
\setcounter{totalnumber}{20}
\renewcommand{\topfraction}{.9999}
\renewcommand{\bottomfraction}{.9999}
\renewcommand{\textfraction}{0}

% source code
\lstset{
basicstyle=\footnotesize,
frame=single,
breaklines=true,
backgroundcolor=\color{light-gray}
}


%##########################################################################
% Abkuerzungen 
\let\abbrev\nomenclature
\renewcommand{\nomname}{Abk"urzungsverzeichnis} 
\setlength{\nomlabelwidth}{.24\hsize} % Punkte zw. Abkrzung und Erklrung
\renewcommand{\nomlabel}[1]{#1 \dotfill}
%\setlength{\nomitemsep}{-\parsep} % Zeilenabstnde verkleinern
\makenomenclature 
% einfuegen mit \abbrev{Abkuerzung}{Beschreibung}


%###########################################################################
% Bearbeitung von einzelnen Kapiteln

%\includeonly{berichttitle}
%\includeonly{berichttoc}
%\includeonly{bericht1}
%\includeonly{bericht2}
%\includeonly{bericht3}
%\includeonly{bericht4}
%\includeonly{bericht5}
%\includeonly{berichtapp}
%\includeonly{berichtlof}
%\includeonly{berichtloc}
%\includeonly{berichtlot}
%\includeonly{berichtbib}


%###########################################################################
\begin{document}

% Titelseite einfgen
%###########################################################################
%
%   Titelseite
%
%###########################################################################
\begin{titlepage}
\vspace*{13mm}
\begin{center}
  \vspace{10mm} 
         {\large \hspace{20mm} IPA robot and ipa-apartment manual\\}
  \vspace{10mm}
       {\Large
          \bf
          \hspace{20mm} Manual for\\}
  \vspace{5mm}
       {\Large
          \bf
          \hspace{20mm} working in the IPA environments\\}

  \vspace{80mm}
  \makebox[40mm]{\large \hspace{16mm} Autors: }\makebox[65mm][l]
                                   {\large Florian Weisshardt}
  \makebox[40mm]{}\makebox[65mm][l]{\large Nadia Hammoudeh Garcia}\\
% \makebox[40mm]{}\makebox[65mm][l]{\large Name}\\
  \vspace{10mm}
         {\large \hspace{20mm} Fraunhofer IPA} \\
  \vspace{5mm}
         {\large \hspace{20mm} Institute for Manufacturing Engineering and Automation} \\
         {\large \hspace{20mm} Stuttgart, Germany} \\
  %\vspace{20mm}
  \vfill
         {\large \hspace{20mm} Last modified on \today}
\end{center}
\end{titlepage}

\clearpage
\thispagestyle{empty}
\cleardoublepage
\thispagestyle{empty}\cleardoublepage % Inhalt auf der rechten Seite beginnen

% Raendereinstellungen fuer Doppelseitigen Ausdruck
\evensidemargin=2pt
\oddsidemargin=40pt

% Zeilenabstand strecken
\renewcommand{\baselinestretch}{1.15}\normalsize


\pagenumbering{roman}

% Inhaltsverzeichnis einfgen
\tableofcontents
\thispagestyle{empty}\cleardoublepage

% Abkrzungsverzeichnis einfgen
%\include{berichtnom}
%\thispagestyle{empty}\cleardoublepage

% Kapitel einfgen
\pagenumbering{arabic}
\chapter{IPA apartment rules}
\label{chap:apartment-rules} 

You can always get the latest version of this manual at \footnote{\url{https://github.com/ipa320/setup/blob/master/manual/IPA_manual.pdf}}.

If you have comments, suggestions or would like to add something to the manual, please contact \href{mailto:fmw@ipa.fhg.de}{fmw@ipa.fhg.de}.

\section{robot side}


\subsection{Working on pool pcs}
don't remove any cable
\subsection{connect your own pc}
use plugs for network and power on the tables (never plug or unplug something under the tables)

you can find cables in cabinet nr 3 (see floor plan ADD REFERENCE


\subsection{Leaving the room}
\subsubsection{Move furniture back}
Table, chairs, cupboards

ADD MAP for standard apartment layout

\subsubsection{Tidy up working places}
remove everything from tables of cob-stud-6 to cob-stud-9
Unplug additional network and power cables and put back to cabinet

\subsubsection{dasfa}


\section{student side}
\subsection{Leaving the room}

\section{Floor plan}
ADD FLOOR PLAN for robot side and student side
\chapter{IPA robot rules}\label{chap:robot-rules}
You can always get the latest version of this manual at \footnote{\url{https://github.com/ipa320/setup/blob/master/manual/IPA_manual.pdf}}.

If you have comments, suggestions or would like to add something to the manual, please contact \href{mailto:fmw@ipa.fhg.de}{fmw@ipa.fhg.de}.

%##########################################################
%##########################################################
\section{Robot cob3-3 and cob3-5}

%##########################################################
\subsection{Working with the robots}

\subsubsection{Allocate a robot}
Allocate a robot for you, see \footnote{\url{http://care-o-bot.org/trac/wiki/WorkingWithIPARobots}}. 

\subsection{Pay attention to the safety and usage information}
For safety and usage information, see Care-O-bot manual\footnote{\url{https://github.com/ipa320/setup/raw/master/manual/Care-O-bot_manual.pdf}}.

%##########################################################
\subsection{Having a break}
A break is considered a short interruption, e.g. for lunch time or toilet break, where the robot is still allocated for you and you will continue working with the robot afterwards.

\subsubsection{Press the emergency button}
Whenever you are out of reach of the emergency stop, e.g. leaving the room, press one of the emergency buttons.

%##########################################################
\subsection{Leaving the robot}

\subsubsection{Mount all casings}
Mount all casings of the robot which means green and grey parts from base and head casing.

\subsubsection{Move components in home position}
When finishing your work with the robot, leave the robot components in their default position. The default position is
\begin{itemize}
\item Arm in "folded" position
\item Torso in "home" position
\item Tray in "down" position
\item Hand in "home" position
\end{itemize}

TODO: ADD IMAGE

\subsubsection{Move robot to charging station}
When finishing your work with the robot, leave it next to the charging station which is the "home" position for the navigation.

TODO: ADD IMAGE

\subsubsection{Disconnect power}
Unplug the power cable and shut down the power supply.

\subsubsection{Charge the remote emergency stop}
When finishing your work with the robot, put the remote emergency stop into its charging station.

%##########################################################
\section{Robot raw3-1}

\subsection{Leaving the robot}

disconnect power

\end{document}
%##########################################################################